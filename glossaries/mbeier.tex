\newacronym{db}{DB}{Datenbank}
\newacronym{dbms}{DBMS}{Datenbankmanagementsystem}

\newglossaryentry{webinterface}{
	name={Webinterface},
	description={Ein Web Interface ist ein System, durch welches Anwender mit dem Netz interagieren. Der Begriff Web Interface steht zumeist für grafische Oberflächen}
}

\newglossaryentry{docker}{
	name={Docker},
	description={Docker ist eine Software, welche es ermöglicht Programme in einer abgeschnittenen Umgebung (genannt Container) laufen zu lassen \cite{dockerEngineOverview}. Das Erstellen dieser Umgebung und das Installieren und Laufen des Programms darin, gestaltet sich hierbei sehr einfach}
}

\newglossaryentry{dcompose}{
	name={Docker Compose},
	description={Docker Compose ist eine Erweiterung von Docker  \cite{dockerComposeOverview}. Mit ihr kann man multiple Container gleichzeitig aufbauen, womit es ermöglicht wird komplexe Infrastruktur - wie in Refundable benötigt - einfach aufzubauen, zu reproduzieren und letztlich auf die Computer, auf denen während der Produktion die Infrastruktur laufen wird, zu liefern}
}

\newglossaryentry{relDb}{
	name={relationale Datenbank},
	description={Datenbank, wo ein Typ von Daten durch eine Tabelle repräsentiert wird. Die Anzahl der Spalten ist hierbei für jeden Datensatz konstant}
}

\newglossaryentry{nosqlDb}{
	name={nicht-relationale Datenbank (\textmd{auch} \textit{NoSQL} Datenbank)},
	description={Datenbank die von dem Konzept, dass Daten durch eine Tabelle repräsentiert wird, abweicht. Hierbei gibt es sehr viele verschiedene Ansätze, die von einer Datenbank-Software zur anderen verschieden sind}
}

\newglossaryentry{json}{
	name={JSON},
	description={Bei JSON (JavaScript Object Notation) handelt es sich um ein kompaktes textbasiertes Datenformat, welches für den Datenaustausch zwischen Schnittstellen entwickelt wurde \cite{rfc4627}}
}

\newglossaryentry{yaml}{
	name={YAML},
	description={YAML (YAML Ain't Markup Language) ist eine Markup-Sprache, die zur Beschreibung von Daten genutzt wird. Hierbei zeichnet sich YAML genau dabei aus, dass es nicht nur für die Maschine, sondern auch für den Menschen gut lesbar ist}
}

\newglossaryentry{xml}{
	name={XML},
	description={XML (Extensible Markup Language) ist eine Markup-Sprache, die zur Beschreibung von Daten genutzt wird \cite{xmlStandard}. Wobei die Datenstruktur von XML über ein Schema verifiziert werden kann}
}

\newglossaryentry{konsistenz}{
	name={Konsistenz},
	description={Konsistenz bedeutet die Einhaltung von Regeln. Im Zusammenhang mit Daten in einer Datenbank ist gemeint, dass die Daten die im \gls{dbms} gespeichert sind die vordefinierten Regeln einhalten müssen}
}

\newglossaryentry{verfugbarkeit}{
	name={Verfügbarkeit},
	description={Verfügbarkeit im Zusammenhang mit einem Dienst bedeutet, dass jener Dienst zu einer bestimmten Zeit erreichbar ist. Eine hohe Verfügbarkeit bedeutet, dass der Dienst (fast) immer erreichbar ist}
}

\newglossaryentry{https}{
	name={HTTPS},
	description={Unter HTTPS (Hypertext Transfer Protocol Secure) versteht sich ein Kommunikationsprotokoll, welches über Verschlüsselung durch Zertifikate eine Verbindung zwischen Server und Client im Web sicherer macht}
}

\newglossaryentry{repo}{
	name={Repository},
	description={Ein Repository ist ein Verzeichnis, in welchem Git aktiv als Versionsverwaltung arbeitet}
}

\newglossaryentry{tpp}{
	name={Third Party Packages},
	description={Third Party Packages sind Programme bzw. Teile von Programmen, die von Entwicklern bereits erstellt und programmiert wurden. Sie wurden daraufhin für die Wiederverwendung von anderen Entwicklern veröffentlicht. Der Gedanke dahinter orientiert sich an dem Sprichwort \enquote{Man muss das Rad nicht neu erfinden.}}
}

\newglossaryentry{bash}{
	name={Bash},
	description={Bash ist eine Unix-Shell, welche die Eingabe von Befehlen ermöglicht. Ein Bash-Skript ist im weiteren Sinne eine Ansammlung von solchen Befehlen, wobei der Kontrollfluss mit Hilfe von Verzweigungen gesteuert werden kann}
}

\newglossaryentry{git}{
	name={Git},
	description={Git ist eine Software zur verteilten Versionsverwaltung. Dies bedeutet, dass Code, der sich in einem Repository, einem von Git verwaltetem Ordner, und speziell die Veränderungen im Code erfasst werden und als verschiedene Versionen des Codes gespeichert werden}
}

\newglossaryentry{github}{
	name={GitHub},
	description={GitHub ist ein online Service, welcher Git-Repositories hosted. Hierdurch wird unter anderem Kollaboration im Team, die Verteilung von Programmen, aber auch generell die Versionsverwaltung auf mehreren Geräten ermöglicht}
}

\newglossaryentry{string}{
	name={String},
	description={Ein String ist ein Datenformat, welches in der Informatik Text und Zeichenketten repräsentiert}
}

\newglossaryentry{bool}{
	name={Boolean},
	description={Booleans sind ein Datenformat, welches einen Wahrheitswert, also entweder wahr (true; 1) oder falsch (false; 0), repräsentiert}
}

\newglossaryentry{object}{
	name={Objekt},
	description={Ein Objekt ist eine ungeordnete Ansammlung an verschiedenen benannten Feldern, welche alle eigene Daten speichern können}
}

\newglossaryentry{array}{
	name={Array},
	description={Ein Array ist eine geordnete Ansammlung an Daten}
}

\newglossaryentry{null}{
	name={Null},
	description={Null ist ein Datentyp, welcher die Abwesenheit von Daten beschreibt. Er ist nicht zu verwechseln mit einer 0. Null wird meistens dann genutzt, wenn Daten noch nicht verfügbar sind}
}

\newglossaryentry{jwt}{
	name={JSON Web Token},
	description={Ein JSON Web Token ist ein kompaktes Übertragungsformat für die sichere Übertragung zwischen zwei Punkten. Die Informationen sind dabei signiert und können dadurch verifiziert werden. \cite{rfc7519} }
}