\newacronym{loc}{LoC}{Lines of Code}

\newglossaryentry{designpattern}{
	name={Entwurfsmuster},
	description={Eine Vorlage, wie man ein Programm oder Software intern strukturiert.}
}

\newglossaryentry{mvvm}{
	name={MVVM},
	description={MVVM (Model-View-ViewModel) ist ein Entwurfsmuster, welches eine Variante des MVC-Patterns ist}
}

\newglossaryentry{mvvc}{
	name={MVVC},
	description={MVVC (Model-View-ViewController) ist ein anderer Name für MVVM}
}

\newglossaryentry{mvc}{
	name={MVC},
	description={MVC (Model-View-Controller) ist ein Entwurfsmuster, welches die Logik eines Programms von dem Interface(View) trennt}
}

\newglossaryentry{mv*}{
	name={MV*},
	description={MV* (Model-View-*) ist eine Zusammenfassung aller Model-View Patterns. MVVM und MVC fallen beide in dieses Schema hinein}
}


\newglossaryentry{vue}{
	name={VueJS},
	description={VueJs ist ein JavaScript-Framework entwickelt von Evan You}
}

\newglossaryentry{angular}{
	name={Angular},
	description={Angular ist ein JavaScript-Framework entwickelt von Google}
}

\newglossaryentry{react}{
	name={React},
	description={React ist ein JavaScript-Framework entwickelt von Facebook}
}

\newglossaryentry{opensource}{
	name={Open-Source},
	description={Open-Source bedeutet, dass der Sourcecode öffentlich ist und von Dritten einsehbar ist}
}

\newglossaryentry{axios}{
	name={axios},
	description={axios ist eine Bibliothek, welche HTTP anfragen senden kann}
}

\newglossaryentry{http}{
	name={HTTP},
	description={Unter HTTP (Hypertext Transfer Protocol) versteht man ein zustandsloses Protokoll zur Übertragung von Daten\cite{wiki_http}}
}