\newglossaryentry{frontend}{
	name={Frontend},
	description={Das Frontend ist die Oberfläche einer Website - also das was zu sehen ist}
}

\newglossaryentry{backend}{
	name={Backend},
	description={Das Backend ist die Funktionalität im Hintergrund des \Gls{frontend}s}
}

\newglossaryentry{css}{
	name={CSS},
	description={Cascading Style Sheet. Zum umgestalten und verschönern der Weboberfläche}
}

\newglossaryentry{html}{
	name={HTML},
	description={Hypertext Markup Language. Die Grundbausteine bzw. Struktur der Weboberfläche. Zum Beispiel Text oder eine Eingabefläche}
}

\newglossaryentry{js}{
	name={JavaScript},
	description={Java Script. Bietet einen Rahmen an Funktionalität und Animationen in Verbindung mit \Gls{html} und \Gls{css}}
}

\newglossaryentry{framework}{
	name={Framework},
	description={Kann als Baukasten gesehen werden. Bietet Möglichkeiten um die normalen Vorhergehensweisen zu kürzen bzw. vereinfachen}
}

\newglossaryentry{java}{
	name={Java},
	description={Eine Programmiersprache der Firma Oracle, wird meist für Desktopapplikationen bzw. für das \Gls{backend}}
}

\newglossaryentry{full stack framework}{
	name={Full Stack Framework},
	description={Ein \Gls{framework} für \Gls{backend} als auch \Gls{frontend}}
}

\newglossaryentry{vanilla}{
	name={Vanilla},
	description={Ein anderer Begriff für Basisausführung}
}

\newglossaryentry{webapplikation}{
	name={Webapplikation},
	description={Wie eine Programm auf dem PC, nur dass das Programm nicht auf dem PC installiert wird, sondern im Internet aufgerufen und geladen wird}
}

\newglossaryentry{sass}{
	name={SASS},
	description={SASS oder auch Syntactically Awesome Style Sheets ist eine Skriptsprache, welche auf CSS basiert. Jedoch bietet SASS deutlich mehr Agilität und einen hohen Grad an Automatisierung}
}

\newglossaryentry{cdn}{
	name={CDN},
	description={CDN oder auch Content Delivery Network. Darunter kann man sich eine Online Datenbank vorstellen, welche optionale Komponenten (wie vorgefertigte CSS Skripte) beinhaltet, damit man sie nicht lokal auf dem Server speichern muss}
}


\newglossaryentry{auszeichnungssprache}{
	name={Auszeichnungssprache},
	description={Vereinfacht gesagt, eine Sprache die von verschiedenen Programmen lesbar ist und für die Gliederung und Formatierung von jeglichen Daten zuständig ist}
}

\newglossaryentry{codepen}{
	name={Code-Pen},
	description={Eine Website, auf der man Live Code ausführen und testen kann}
}

\newglossaryentry{tag}{
	name={Tag},
	description={Markiert in HTML ein HTML-Element. Auch als Marken bezeichnet}
}

\newglossaryentry{mockup}{
	name={Mockup},
	description={Ein Design Prototyp einer Website, welcher rein optisch ist und keine Funktionalität besitzt}
}


