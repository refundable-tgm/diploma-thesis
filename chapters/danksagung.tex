%!TEX root=../main.tex
\chapter{Danksagung} 
Der fünfte Jahrgang, der Maturajahrgang, ist keine einfache Zeit. Er ist für uns Schüler voller Herausforderungen. Seien es die beiden normalen Semester, die es zu absolvieren gilt, die Diplomarbeit oder die Vorbereitung auf die Matura. Danach gilt es bei der Matura selbst noch einmal alles zu geben und das unter Beweis zu stellen, was man die mindestens 13 Jahre lange Schullaufbahn über erlernt und geübt hat. Zusätzlich zu diesen Herausforderungen erschwert die Coronavirus-Pandemie dieses Jahr den schulischen Alltag massiv. Gerade deswegen wollen wir hiermit jenen ausdrücklich von Herzen danken, die uns in dieser schwierigen Zeit geholfen haben und zur Seite gestanden sind.\\

Zuerst wollen wir uns bei unserem Diplomarbeitsbetreuer Herrn Professor Zakall und unserem Betreuer aus dem ITP-Unterricht Herrn Professor Dolezal dafür bedanken, dass sie jederzeit ein offenes Ohr für unsere Probleme hatten und auch immer mit Rat und Hilfe uns bei der Lösung unserer Probleme beigestanden sind. Des Weiteren wollen wir Herrn Professor Borko für die Organisation des Diplomarbeitsseminars und für die immer klaren Antworten auf all unserer unzähligen Fragen bedanken.\\

Dieser Maturajahrgang war auf Grund der angehenden Pandemie wahrscheinlich einer der Schwierigsten und Aufwendigsten der letzten Jahren. Von Lockdown zu Lockdown wurden sozialen Kontakte eingeschränkt und wir konnten mit unseren Freunden und Klassenkollegen nur online reden. Aus unserer Situation heraus wollen wir auch unseren Freunden in der Klasse danken, die uns durch diese schwere Zeit geholfen haben. Wir wollen hier speziell Maximilian Frühmann für die vielen Ratschläge, die er uns gegeben hat, und für das offene Ohr für etwaige Probleme bedanken. Ein weiterer Dank gilt Tobias Schrottwieser für die Aufheiterungen egal in welcher Situation und dass er uns auf bessere Gedanken gebracht hat, danken. Diese zwei besonders und einige weitere seid ein unersetzbarer Teil unseres Freundeskreises und Klassenkollegen, die sich andere nur wünschen können.\\

Zuletzt wollen wir auch unseren Familien danken, die in diesen turbulenten Zeiten immer für uns da waren und an unserer Seite standen. Wir glauben, dass, wenn die Pandemie endlich vorüber ist, wir mit neuen Fähigkeiten daraus herausgehen werden, und dies nur auf Grund jener Förderung im Leben, die wir von unseren Familien erhalten haben. 
Speziell jenen, die unseren Abschluss leider nicht mehr miterleben können, wollen wir von ganzem Herzen für die unendliche Inspiration danken. Wenn wir eines mit Sicherheit wissen, dann, dass ihr immer stolz auf uns wart und das auch immer sein werdet.