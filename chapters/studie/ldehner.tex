%!TEX root=../../main.tex

\section{Frontend - responsives Webdesign}
\label{chapter:study-frontend}
	\subsection{Einleitung}
	\label{chapter:study-frontend-einleitung}
	In diesem Kapitel werden die Anforderungen des \textit{\Gls{frontend}-Deisgns} geschildert. Da es mehrere Möglichkeiten gibt, das \textit{Frontend} zu realisieren, werden hier drei wesentliche Methoden verglichen. Die erste Variante wäre, ganz klassisch \textit{\Gls{html}}, \textit{\Gls{css}} und \textit{\Gls{js}} zu verwenden. Die zweite Methode, die zum Vergleich herangezogen wird, verwendet statt \textit{\Gls{vanilla} CSS} die etwas agilere Sprache \Gls{sass}. Die dritte und auch letzte Methode in diesem Vergleich ist, die Arbeit durch die Verwendung eines \textit{CSS \Gls{framework}s} zu vereinfachen. Anschließend werden die drei verschiedenen Methoden gegenübergestellt und die, für unser Projekt \textit{Refundable} am besten geeignete Variante ausgewählt. Zu guter Letzt wird das \textit{Design} im Hinblick auf die Zielgruppe der Lehrer analysiert, die sich möglichst gut und schnell auf der Webseite zurecht finden soll.
	
	\subsection{HTML, CSS, JS}
	\label{chapter:study-frontend-html-css-js}
	Der eigentliche Standard \textit{HTML5} wird in der Praxis meist als Überbegriff für \textit{HTML}, \textit{CSS} und \textit{JS} verwendet \cite{html5-css3-handbuch}. In den folgenden Kapiteln wird erklärt, wozu \textit{HTML}, \textit{CSS} und \textit{JS} da sind und welche Funktionalitäten sie bieten.
	
		\subsubsection{HTML}
		\label{chapter:study-frontend-html}
		\textit{HTML} ist eine \Gls{auszeichnungssprache}, sie steht für \textit{\enquote{Hypertext Markup Language}} und wurde 1989 von dem britischen Informatiker Tim Burners-Lee veröffentlicht.\\ 
		\begin{center}
			\textit{\enquote{\textit{Hypertext} bezeichnet die Möglichkeit, Texte mit Hilfe von Hyperlinks, oder kurz Links, miteinander zu verbinden}}\cite{html5-css3-def}.
		\end{center}
		Dies heißt, dass man mittels \textit{Hypertexts} (Links) auf der Seite beliebig zwischen Sektionen hin und her springen kann.
		Auszeichnungssprachen werden im Fachjargon auch als \textit{Markup Language (ML)} bezeichnet \cite{auszeichnungssprachen}. \textit{Markup Languages} werden in zwei verschiedene Gruppen aufgeteilt, zum einen \textit{Procedural Markup Languages (PML)}, das sind jene Auszeichnungssprachen, die für die Verarbeitung von Daten optimiert sind. Zum anderen \textit{Descriptive Markup Languages (DML)}, diese sind für die logische Strukturierung von Daten da.\\Bekannte Beispiele hierfür sind:
		\begin{itemize}
		\item PML
		\subitem PDF
		\subitem TeX
		\item DML
		\subitem HTML
		\subitem SVG
		\end{itemize}
	\captionof{listing}{PML/DML Beispiele}
	\label{list:dmlbsp}~\\
		\textit{HTML} wird hauptsächlich verwendet, um Texte, Grafiken und \textit{Hyperlinks} (Links) darzustellen \cite{html5-css3-handbuch, html5-css3-def}. Die Bearbeitung von \textit{HTML-Dokumenten} ist relativ einfach und unkompliziert, da es eine rein textbasierte Sprache ist und mit jedem Texteditor bearbeitet werden kann.\\
		Allerdings ist \textit{HTML} nicht mit einer Programmiersprache zu verwechseln, da nur \textit{\Gls{tag}s} und keine Befehle oder Anweisungen verwendet werden. Solche \textit{Tags} können wie folgt aussehen:
		\begin{code}{html}
			<tagname> Inhalt</tagname>
			<einzeltag attribut="123">
		\end{code}
	\captionof{listing}{\textit{HTML Tags}}
	\label{list:htmltags} ~\\
		Das grundlegende Gerüst von \textit{HTML} besteht aus einer Deklaration von \textit{HTML}, einem \textit{html-}, \textit{head-} und \textit{body-Tag}. In den \textit{html-Tag} kommt ein \textit{title-Tag}, in diesem wird der Titel der Webseite angegeben und ein \textit{meta-Tag}, in diesem werden Meta-Informationen angegeben. In den \textit{Body-Tag} kommen wiederum \textit{Tags}, die den Inhalt der Seite darstellt. 
		\begin{code}{html}
			<!DOCTYPE html>
			<html lang="de">
				<head>
					<meta charset="UTF-8">
					<meta name="viewport" content="width=device-width, initial-scale=1.0">
					<title>Kurzes BSP</title>
				</head>
				<body>
					<h1>Überschrift</h1>
					<button>Drück mich!</button>
				</body>
			</html>
		\end{code}
	\captionof{listing}{Kurzes \textit{HTML} Beispiel}
	\label{list:htmlbsp} ~\\
		Wenn man die Datei nun im \textit{Browser} öffnet, sieht dies wie folgt aus:
		\begin{figure}[H]
			\centering
			\includegraphics[width=0.2\linewidth]{images/ldehner_study/html1}
			\caption[\textit{HTML} Beispielseite]{Beispiel einer \textit{HTML} Seite mit einer Überschrift und einem Button}
			\label{fig:htmlbsp}
		\end{figure}
		~\\
		Da \textit{HTML} nur für die Grundstruktur einer Webseite gedacht ist, also zum Beschreiben der Struktur und des Inhalts, ist das \textit{Design} noch nicht sonderlich ansprechend. Um dies zu verändern, wird \textit{CSS} benötigt.
		
		\subsubsection{CSS}
		\label{chapter:study-frontend-css}
		\textit{Cascading Stylesheets}, oder kurz \textit{CSS}, ist für den \textit{Style}, also für das Aussehen der Webseite verantwortlich. Da \textit{HTML} anfangs nur im Printbereich verbreitet wurde, war es nicht notwendig, die Seiten zu gestalten \cite{html5-css3-def, html5-css3-handbuch}. Das Internet bekam aber einen immer stärker werdenden Einfluss und daher auch eine höhere Bekanntheit. Deswegen wurde \textit{HTML} mit der Formatierungssprache \textit{CSS} ergänzt.\\
		Mittels \textit{CSS} sind unter anderem folgende Dinge möglich:
		\begin{itemize}
			\item Den Hintergrund verändern
			\item Die Schrift verändern
			\item Die Webseite automatisch an die Bildschirmgröße anpassen
			\item Den Formfaktor von Elementen verändern
			\subitem Größe
			\subitem Rand
			\subitem Farbe
			\subitem Form
			\subitem Schatten
			\subitem Hover-Effekt
		\end{itemize}
	\captionof{listing}{Beispiele Verwendung von \textit{CSS}}
	\label{list:bspcss}~\\
		Man kann \textit{CSS} auf verschiedene Arten in \textit{HTML} benützen. Als Beispiel werden wir eine Überschrift in die Mitte der Webseite setzen, die Schriftgröße auf \enquote{40pt} stellen und die Schriftart auf \enquote{sans-serif} ändern.\\
		Die Umsetzung kann auf mehrere Arten erfolgen. Beispielsweise fügt man einem Element ein \textit{Style-Attribut} hinzu und ändert direkt im Element das Aussehen. Hierbei ist darauf zu achten, dass nur das Element, in dem man diese Änderungen vornimmt, verändert wird:
		\begin{code}{html}
			<h1 style="text-align: center; font-size: 40pt; font-family: sans-serif;">Ich bin eine tolle Überschrift</h1>
		\end{code}
	\captionof{listing}{Beispiele Verwendung von Inline-CSS}
	\label{list:bspinlinecss}~\\
		Man kann auch im \enquote{head} einen \textit{style-Bereich} eröffnen und dort das Aussehen verändern. Dabei ist darauf zu achten, dass man den \textit{Style} von allen Elementen mit dem Tag, den man ausgewählt hat, verändert. Um dies zu verhindern, kann man Elementen auch eine \textit{ID} (für einzelne Elemente verwendbar) oder eine \textit{CLASS} (für mehrere Elemente verwendbar) hinzufügen, sowie diese im \textit{CSS-Code} auswählen und verändern:
		\begin{code}{html}
				<head>
					<style>
					//Für das ganze Element
					h1 {
						text-align: center; 
						font-size: 40pt; 
						font-family: sans-serif;
					}
				
					//Für IDs
					#ueberschrift1 {
						text-align: center; 
						font-size: 40pt; 
						font-family: sans-serif;
					}
					
					//Für Klassen
					.ueberschriftenGruppe {
						text-align: center; 
						font-size: 40pt; 
						font-family: sans-serif;
					}
					</style>
				</head>
		\end{code}
	\captionof{listing}{Beispiel CSS}
	\label{list:cssbsp} ~\\
		Ebenfalls kann man ein \textit{CSS-File}, welches den Inhalt des obigen \textit{style-Tags} hat, im \textit{head} als externes File einbinden:
		\begin{code}{html}
			<head>
				<link rel="stylesheet" href="file.css" type="text/css">
			</head>
		\end{code}
	\captionof{listing}{CSS Verlinkung}
	\label{list:csslink} ~\\
		Wie man erkennen kann, ist dem ganzen kein Ende gesetzt und mit viel Aufwand kann man alles Denkbare verändern. Um den Aufwand jedoch gering zu halten, sind \textit{SASS} und \textit{CSS-Frameworks} da, welche in \autoref{chapter:study-frontend-sass} und \autoref{chapter:study-frontend-frameworks} erklärt werden.
		\subsubsection{Java Script}
		\label{chapter:study-frontend-js}
		\textit{JavaScript} wird verwendet, um Elementen Funktionen zu geben. Zum Beispiel, dass wenn man auf einen \textit{Button} drückt, ein Fenster \textit{aufpoppt}, ein neues Element hinzugefügt wird, oder ein Element im Nachhinein verändert wird. Da \textit{JavaScript} eine Skriptsprache ist, kann man auch eigene Funktionen schreiben und Variablen verwenden. Für das \textit{Designen} braucht man \textit{JS} nur, wenn man mit \textit{CSS} oder \textit{SASS} arbeitet. Wird ein \textit{Framework} verwendet, hat dieses meist ein \textit{JavaScript-Framework} inkludiert und man muss kein \textit{JavaScript} mehr verwenden.
	\subsection{SASS}
	\label{chapter:study-frontend-sass}
	\textit{Syntactically Awesome Style Sheets} oder kurz \textit{SASS} ist eine Erweiterung von \textit{CSS} \cite{jump-start-sass}. Es fügt \textit{CSS} ein paar Funktionalitäten von \textit{Java Script} hinzu. \textit{SASS} wird aber nicht wie \textit{CSS} direkt in das \textit{HTML-File} eingebunden und kann auch nicht direkt in ein Element geschrieben werden. Das \textit{.sass} File muss erst kompiliert werden. Anschließend wird ein \textit{.css} \textit{File} generiert, welches man im \textit{HTML-Code} einbinden kann. Man kann unter anderem Funktionen erstellen, um Elemente zu verändern. Zusätzlich kann man Variablen kreieren und so zum Beispiel eine Primärfarbe festlegen, wodurch man sich nicht immer den \textit{HEX-Code} von einer bestimmten Farbe merken muss. Ebenfalls kann man durch die Variablen die Farbe von mehreren Elementen auf einmal ändern. Dadurch kann man zum Beispiel \textit{Light}- und \textit{Darktheme} in die Webseite einbauen.

%	\begin{code}{sass}
	%	$standard-width: 200px
	%	$standard-color: #03f0fc
	%	
	%	.button {
	%		background-color: $standard-color;
	%		width: $standard-width;
	%	}
	%
	%	.button big{
	%		background-color: $standard-color;
	%		width: multiply($standard-width, 2);
	%	}
	%
	%	@function multiply($a, $b) {
	%		@return ($a * $b);
	%	}
%	\end{code}
	
	
	\subsection{Frameworks}
	\label{chapter:study-frontend-frameworks}
	\textit{Frameworks} für \textit{HTML}, \textit{CSS} und \textit{JS} beinhalten vorgestaltete Komponenten, welche mittels Klassen, einem \textit{HTML} Element hinzugefügt werden. Da \textit{Frameworks} das Arbeiten an einer Webseite wesentlich einfacher machen und wir dieses Hilfsmittel auch benutzen werden, müssen folgende Fragen beantwortet werden, um das optimale Framework für dieses Projekt auszuwählen:
	\begin{itemize}
		\item \textbf{Welche \textit{CSS-Frameworks} unterstützen explizit die \textit{User-Experience} und \textit{Usability}?}
		\item \textbf{Welche Vor- und Nachteile bringen diese \textit{Frameworks} mit sich?}
	\end{itemize}
\captionof{listing}{Fragestellungen - Frameworks}
\label{list:fragenframeworks} ~\\
	Zum Vergleich werden vier sehr verbreitete und geschätzte \textit{Frameworks} herangezogen:
		\subsubsection{Bootstrap}
		\label{chapter:study-frontend-frameworks-bootstrap}
		\textit{Bootstrap} ist ein \textit{Framework}, welches sowohl am bekanntesten ist, als auch am meisten verwendet wird \cite{introduction-bootstrap, learning-bootstrap}. Das von \textit{Twitter} entwickelte \textit{Framework} hat in der aktuellen Version 4.5 viele verschiedene Komponenten, die optimal für erfahrene Entwickler, aber auch für Anfänger geeignet sind. In der Dokumentation von \textit{Bootstrap} gibt es einige \textit{Templates} und sehr viel gut dokumentierten Beispiel Code\cite{introduction-bootstrap}. \textit{Bootstrap} kann über einen \textit{Paketmanager}, ein \textit{CDN}, oder als heruntergeladene Datei zur \textit{HTML-Datei} hinzugefügt werden \cite{bootstrap-docu}. Durch diesen großen Verwendungsgrad gibt es viele Erweiterungen und zahlreiche Plattformen auf denen man sich informieren kann, falls man Probleme bei der Entwicklung der Webseite hat \cite{learning-bootstrap}. \textit{Bootstrap} bietet, mit seinem überschaubaren \textit{Grid-System}, eine gute Umsetzung für die Responsivität der Seite, damit sie auf allen möglichen Endgeräten perfekt aussieht.
		\paragraph{Pro / Contra}
		\subparagraph{Pro}
		\begin{itemize}
			\item \textit{HTML/CSS/JS Framework}
			\item Umfangreich
			\item Mittels \textit{SASS} veränderbar
			\item Zahlreiche Erweiterungen
			\item Viele Erkundungsmöglichkeiten
			\item Viele Einbindungsmöglichkeiten
			\item Arbeitet gut mit \textit{VueJS} zusammen
			\item Responsiv
		\end{itemize}
	\captionof{listing}{Bootstrap Pro}
	\label{list:bootstrappro}
		\subparagraph{Contra}
		\begin{itemize}
			\item Viele Webseiten sehen gleich aus
			\item Komplexer zu erlernen
			\item Kein schönes Standarddesign
		\end{itemize}
	\captionof{listing}{Bootstrap Contra}
	\label{list:bootstrapcontra}
	
		\subsubsection{Materialize}
		\label{chapter:study-frontend-frameworks-materialize}
		\textit{Materialize} ist, wie \textit{Bootstrap}, ein intuitives \textit{HTML}, \textit{CSS} und \textit{JS Framework} \cite{materialize-intro}. Es ist \textit{Bootstrap} sogar sehr ähnlich, aber simpler aufgebaut. Dadurch hat es auch nicht so viele verschiedene Komponenten und nicht so viel Beispiel Code wie \textit{Bootstrap}. \textit{Materialize} legt vor allem Wert auf das \textit{Design}, welches von\textit{ Google's Material Design} abstammt. Ein weiterer großer Punkt ist \textit{Usability} und \textit{User Experience}, welche unter anderem Hand in Hand mit dem \textit{Design} gehen. Um die \textit{User Experience} möglichst hoch zu halten, arbeitet \textit{Materialize} viel mit Animationen. \textit{Materialize} kann ebenfalls über ein \textit{\Gls{cdn}}, \textit{Paketmanager} oder als Datei in das Projekt eingebunden werden. Falls man das \textit{Design} verändern will, stellt \textit{Materialize} ebenfalls noch eine \textit{SASS} Version zur Verfügung \cite{WebDocMaterialize}.
		\paragraph{Pro / Contra}
		\subparagraph{Pro}
		\begin{itemize}
			\item HTML/CSS/JS Framework
			\item Mittels SASS veränderbar
			\item Ansehnliches Standard-Design 
			\item Viele Animationen
			\item Viele Einbindungsmöglichkeiten
			\item Einfach zu verstehen
			\item Responsiv
		\end{itemize}
	\captionof{listing}{Materialize Pro}
	\label{list:materializepro}
		\subparagraph{Contra}
		\begin{itemize}
			\item Einige Probleme mit VueJS
			\item Weniger Erkundungsmöglichkeiten
			\item Beispielcode ist oft unvollständig
		\end{itemize}
	\captionof{listing}{Materialize Contra}
	\label{list:materializecontra}
	
		\subsubsection{ZURB Foundation}
		\label{chapter:study-frontend-frameworks-foundation}
		\textit{Foundation} wird, wie \textit{Bootstrap} und \textit{Materialize}, als intuitives \textit{Web-Framework} bezeichnet \cite{foundation-intro}. Das von \textit{Zurb} entwickelte \textit{Framework} beinhaltet viele verschiedenen Komponenten, die für mobile Endgeräte optimiert sind. Mit dem \textit{Framework} können Webseiten schnell und effizient gestaltet werden. Darum ist es auch in zwei verschiedene Kategorien geteilt, in das \textit{Framework} für das Web und in das \textit{Framework}, welches explizit für \textit{HTML-Mails} gedacht ist. In der \textit{Complete-Version} sind alle Komponenten enthalten. In der \textit{Essential-Version} sind nur die essentiellen Komponenten, wie zum Beispiel \textit{Buttons}, enthalten. Man kann sich seine \textit{ZURB-Datei} auch mit allen Komponenten, die man benötigt, selbst konfigurieren. Auch eine \textit{SASS} Version ist verfügbar, falls man nur das Aussehen der Komponenten verändern möchte. Natürlich ist auch eine \textit{CDN} Einbindung möglich, die für die schnellste Ladegeschwindigkeit optimiert ist.
		\paragraph{Pro / Contra}
		\subparagraph{Pro}
		\begin{itemize}
			\item \textit{HTML/CSS/JS Framework}
			\item Mittels \textit{SASS} veränderbar
			\item Auf Bedürfnisse anpassbar
			\item Ansehnliches Standard-\textit{Design} 
			\item Viele Einbindungsmöglichkeiten
			\item Einfach zu verstehen
			\item Responsiv
		\end{itemize}
	\captionof{listing}{ZURB Foundation Pro}
	\label{list:zurbpro}
		\subparagraph{Contra}
		\begin{itemize}
			\item Keine persönliche Erfahrung
			\item Relativ junges Framework
			\item Nicht so viele Erkundungsmöglichkeiten
			\item Kein ansprechendes Standarddesign
		\end{itemize}
	\captionof{listing}{ZURB Foundation Contra}
	\label{list:zurbcontra}

	\subsection{Vergleich}
	\label{chapter:study-frontend-vergleich}
	Der Vergleich der \textit{Frameworks} setzt sich aus dem Umfang, den Erweiterungen, dem Aussehen, der Leichtigkeit im Bezug auf das Erlernen des \textit{Frameworks}, dem Umfang der Dokumentation, der Kompatibilität mit \textit{VueJS} und unserer Erfahrung, mit dem jeweiligen \textit{Framework} zusammen.\\
	Die Punkte (0-9) setzen sich immer in der Relation mit den anderen \textit{Frameworks} zusammen, wobei 9 die Beste, also maximale Punktzahl ist. Die Basis 0 ist in diesem Vergleich mit den Funktionen und Aufwand von \textit{Vanilla CSS} gleichzusetzen. Alle dafür benötigten Informationen wurden von der jeweiligen Dokumentationsseite bezogen.
	~\\
	\captionof{table}[Vergleich \textit{HTML Frameworks}]{Vergleich zwischen\textit{ Bootstrap, Materialize} und \textit{Foundation}}\label{tbl:comparison}
	\begin{center}
		\begin{table}
			\centering
		\begin{tabular}{rccc}
			\hline
			\multicolumn{1}{|r|}{{\underline{\textbf{Kriterien}}}} & \multicolumn{1}{c|}{{\underline{\textbf{Bootstrap}}}} & \multicolumn{1}{c|}{{ \underline{\textbf{Materialize}}}} & \multicolumn{1}{c|}{{\underline{\textbf{Foundation}}}} \\ \hline
			\multicolumn{1}{|r|}{\textbf{Umfang}}          & \multicolumn{1}{c|}{9}                        & \multicolumn{1}{c|}{9}                          & \multicolumn{1}{c|}{7}                         \\ \hline
			\multicolumn{1}{|r|}{\textbf{Erweiterungen}}   & \multicolumn{1}{c|}{9}                        & \multicolumn{1}{c|}{3}                          & \multicolumn{1}{c|}{7}                         \\ \hline
			\multicolumn{1}{|r|}{\textbf{Aussehen}}        & \multicolumn{1}{c|}{6}                        & \multicolumn{1}{c|}{9}                          & \multicolumn{1}{c|}{5}                         \\ \hline
			\multicolumn{1}{|r|}{\textbf{Leichtigkeit}}    & \multicolumn{1}{c|}{6}                        & \multicolumn{1}{c|}{9}                          & \multicolumn{1}{c|}{8}                         \\ \hline
			\multicolumn{1}{|r|}{\textbf{Dokumentation}}   & \multicolumn{1}{c|}{9}                        & \multicolumn{1}{c|}{6}                          & \multicolumn{1}{c|}{9}                         \\ \hline
			\multicolumn{1}{|r|}{\textbf{Kompatibilität}}  & \multicolumn{1}{c|}{9}                        & \multicolumn{1}{c|}{5}                          & \multicolumn{1}{c|}{7}                         \\ \hline
			\multicolumn{1}{|r|}{\textbf{Erfahrung}}       & \multicolumn{1}{c|}{5}                        & \multicolumn{1}{c|}{9}                          & \multicolumn{1}{c|}{0}                         \\ \hline
			\textbf{Gesamt}                                & 53                                            & 50                                              & 43                                            
		\end{tabular}
	\end{table}
	\end{center}
\subsubsection{Umfang}
\label{chapter:study-frontend-vergleich-umfang}
Die Punktevergabe des Umfangs setzt sich zusammen aus der Anzahl der Komponenten und Hilfsklassen. \textit{Bootstrap} und \textit{Materialize} haben in etwa die selbe Anzahl (53 \& 55) und bekommen daher die Maximalpunktzahl \cite{bootstrap-docu,WebDocMaterialize,foundation-webdocu}. \textit{Foundation} hat hingegen deutlich weniger Komponenten (41) als die anderen zwei \textit{Frameworks} und hat daher nur 7 Punkte bekommen.
\subsubsection{Erweiterungen}
\label{chapter:study-frontend-vergleich-erweiterungen}
\textit{Bootstrap} hat durch das langjährige Bestehen und durch die große Reichweite unzähliger Erweiterungen, die Schwächen, wie zum Beispiel das Standarddesign wieder auszumerzen. Deswegen hat \textit{Bootstrap} hier auch die volle Punkteanzahl bekommen \cite{introduction-bootstrap}. \textit{Materialize} hat nur kostenpflichtige, offizielle Plugins, deswegen hat dieses \textit{Framework} nur 3 Punkte bekommen. \textit{Foundation} hingegen, hat wiederum eine ausreichende Auswahl an \textit{Plugins}, die sogar in der Dokumentation verlinkt sind \cite{foundation-webdocu}. 
\subsubsection{Aussehen}
\label{chapter:study-frontend-vergleich-aussehen}
Bei dem Aussehen des \textit{Standard-Designs} gewinnt klar und deutlich \textit{Materialize}, da es die ansprechendste Darstellung mit einem schönen \textit{Design} bietet \cite{WebDocMaterialize}. \textit{Bootstrap} und \textit{Foundation} legen mehr Wert auf die Funktionalität des \textit{Frameworks} und weniger auf das Aussehen \cite{bootstrap-docu, foundation-webdocu}. Dies merkt man daran, dass die Elemente optisch nur minimal von den \textit{HTML}-Standardelementen abweichen. \textit{Materialize} hat hingegen ein komplett neues und modernes \textit{Design} aufgezogen.
\subsubsection{Leichtigkeit}
\label{chapter:study-frontend-vergleich-leichtigkeit}
Aus eigener Erfahrung kann das Team sagen, dass \textit{Materialize} wirklich einfach zu erlernen ist \cite{WebDocMaterialize}. Bei einem groben Einlernen in \textit{Foundation} hat das Team festgestellt, dass es dem Aufbau von Materialize sehr ähnelt. \textit{Bootstrap} hingegen sah auf den ersten Blick sehr kompliziert aus und war am Anfang sehr schwer zu verstehen. Nach ein paar Stunden Einlernen war schon ein sehr guter \textit{Workflow} vorhanden.
\subsubsection{Dokumentation}
\label{chapter:study-frontend-vergleich-doku}
Auf den ersten Blick erscheint die Dokumentation von \textit{Materialize} als sehr gut, jedoch ist der Beispielcode zu einzelnen Komponenten in manchen Fällen unvollständig und man muss sich mit mühsamer Recherche nach Lösungen erkunden \cite{WebDocMaterialize}. \textit{Bootstrap} und \textit{Foundation} sind sehr gut dokumentiert, haben viele Code-Beispiele und sind in manchen Fällen sogar mit \Gls{codepen} Beispielen bestückt \cite{bootstrap-docu, foundation-webdocu}.

\subsubsection{Kompatibilität}
\label{chapter:study-frontend-vergleich-kompatibilität}
Wie in \autoref{chapter:study-datenschnittstelle} erklärt wird, werden wir als \textit{JS Framework} \textit{VueJS} verwenden. Kompatibel sind alle \textit{Frameworks}, jedoch ist \textit{Bootstrap} für die Zusammenarbeit mit \textit{Vue} optimiert \cite{bootstrap-docu}. Da das Projektteam in einem früheren Projekt bereits \textit{Materialize} mit \textit{Vue} benutzt hat, ist bekannt, dass in Verbindung der beiden \textit{Frameworks} Probleme auftreten können. \textit{Foundation} ist nicht speziell dafür optimiert mit \textit{Vue} zusammenzuarbeiten. Die Funktionalität ist trotzdem vorhanden. Jedoch wurde \textit{Foundation} bis jetzt von keinem Mitglied der Projektes in Verbindung mit \textit{Vue} verwendet.

\subsubsection{Erfahrung}
\label{chapter:study-frontend-vergleich-erfahrung}
Das\textit{ Frontend-Team} hat bis jetzt hauptsächlich Erfahrung mit \textit{Materialize} gemacht. \textit{Bootstrap} wurde in der Schule kurz angeschnitten, aber im privaten Umfeld etwas vertieft. Mit \textit{Foundation} hingegen wurde noch keine Erfahrung gesammelt.

\subsection{Fazit}
\label{chapter:study-frontend-vergleich-fazit}
Nach dem Vergleich, der in dieser Arbeit durchgeführt wurde, ist es leicht, die optimale Auswahl der Umsetzung für \textit{Refundable} zu treffen. Eine Entwicklung ohne ein \textit{Framework} kommt auf Basis der Analyse nicht in Frage, da der Aufwand für die Umsetzung wesentlich höher wäre. Wir haben uns schließlich für \textit{Bootstrap} entschieden, da es im Vergleich eindeutig am besten abschneidet und auch am Besten mit \textit{VueJS} harmoniert.

