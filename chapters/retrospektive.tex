%!TEX root=../main.tex
\chapter{Retrospektive}
\section{Probleme}
Die Zusammenarbeit zwischen den Teammitgliedern war aufgrund der derzeitigen Pandemie stark beeinträchtigt, wodurch es viele neue Herausforderungen zu meistern gab.\\

Durch diese Situation wurde die Entwicklung der Datenschnittstelle deutlich erschwert und sie konnte aufgrund fehlender Informationen erst sehr spät entwickelt werden. Auch die Entwicklung der Webseitenlogik hat durch fehlende Kommunikation länger gedauert als geplant. Dadurch wurden Fehler in der Navigation, sowie bei den Eingaben des Nutzers einprogrammiert, welche später behoben werden mussten.\\

Durch herausragende Zusammenarbeit gegen Ende des Projekts konnten jedoch alle Herausforderungen bewältigt werden.
\section{Erkenntnisse}
Die Kommunikation ist ein essenzieller Teil eines Projektes, welcher auf gar keinen Fall vernachlässigt werden darf. Des Weiteren hat das Projektteam gelernt, sich nicht durch Rückschritte frustrieren zu lassen, sondern umso mehr daran zu arbeiten, dass das Endprodukt funktionsfähig ist.
\newpage
\section{Webdesign}
Alle erforderlichen Seiten des \textit{Frontends} wurden erfolgreich implementiert. Die Webseite wurde mittels \textit{Bootstrap}-Komponenten befüllt, welche im Nachhinein an das \textit{Design} von \textit{Refundable} angepasst wurden. Alle benötigten Formulare wurden in \textit{HTML}-Form kreiert und sind für den Nutzer mit ausreichend Informationen und Hinweisen bestückt. Auch die Übersicht der normalen und administrativen Nutzer wurde vollständig erstellt.\\
~\\
Jedoch hat das Projektteam viel aus der Arbeit im Projekt gelernt, und hat deshalb folgende Verbesserungsvorschläge entworfen:
\begin{itemize}
	\item Das Design der \textit{Login-} und \textit{Startseite} auf den restlichen Seiten stärker fortführen
	\item Die Darstellung auf mobilen Geräten verbessern
	\item Das Reisekostenformular verschönern
\end{itemize}
\captionof{listing}[Auflistung von Verbesserungsvorschlägen]{Auflistungen von Verbesserungsmöglichkeiten}

\newpage

\section{Backend und Infrastruktur}

Sowohl das \textit{Backend}, als auch die Infrastruktur, wurden komplett implementiert. Die Hauptfunktionalität wurde fertig programmiert und in das System und in die Betriebsumgebung eingepasst. Zusätzlich wurde auch die Infrastruktur entworfen und dynamisch \textit{deployable} designt, sodass dieses einfach aufbaubar und steuerbar ist. Die \textit{REST}-Schnittstelle bereitet Daten entsprechend auf und stellt sie auf Anfrage eines \textit{Clients} zur Verfügung. Auch die Schnittstellenherausforderungen im \textit{Backend} mit Diensten und Protokollen wie \textit{LDAP} oder \textit{Untis} wurden gelöst und auf eine Art und Weise, dass die Implementierung klar und effizient ablaufen kann.\\

Jedoch gibt es immer meistens bei Softwareprojekten mit klaren \textit{Deadlines} immer noch etwas, das noch implementiert, verbessert oder neu entworfen werden kann. So ist es auch hier im Backend und in der Infrastruktur. Aus dieser Erkenntnis heraus wurde eine Liste erstellt, welche weiteren Implementierungen und möglichen Verbesserungen wichtig für das System wären und die Software unterstützen könnten:

\begin{itemize}
	\item Versenden von Mails (bei Statusänderungen oder als Erinnerungen)
	\item Ein umfangreichendes \textit{Logging}-System, um jegliche Aktionen aufzeichnen zu können
	\item Ein einfacheres Design des Datenmodells entwerfen
	\item PDF-Vorlagen besser und übersichtlicher designen
	\item Die \textit{REST}-Schnittstelle mit weiteren Steuerungsmöglichkeiten und kleineren \textit{Endpoints} zur Zustandsänderung ausstatten, wodurch große \textit{Endpoints} (wie beispielsweise jener um Anträge zu bearbeiten) obsolet werden würden und die \textit{Endpointstruktur} klarer und übersichtlicher werden würde.
	\item modulare Wahl zwischen Datenbanken inklusive Datenbankschnittstelle über \textit{Strategy-Pattern} implementieren, und über Nutzerkonfiguration bei der Installation Entscheidung ermöglichen.
	\item Grenzwertüberprüfungen im \textit{Backend} einbauen, sodass nur letztlich valide Werte genutzt werden können
	\item Reduzierung der \textit{McCabe-Metrik} (zyklomatische Komplexität)
	\item Generelle \textit{Performance} Verbesserungen
	\item Weitere Sicherheitsmaßnahmen entwerfen und implementieren (beispielsweise HTTPS Implementierung für \textit{Webserver} und \textit{REST}-Schnittstelle)
\end{itemize}
\captionof{listing}[Auflistung von weiteren Ideen]{Auflistungen von Ideen zur weiteren Implementierung und Verbesserung}
\newpage
\section{Datenschnittstelle und Webseitenlogik}
Die Webseitenlogik wurde vollends implementiert und lässt den Benutzer auf jede gewünschte Seite zugreifen. Die Datenschnittstelle ist ebenfalls implementiert. Die Datenschnittstelle ist so konfiguriert, dass eine \textit{REST}-Schnittstelle angesprochen wird, um die Daten eines Benutzers zu erhalten, damit dieser das System verwenden kann. Die Rechte, die bei jedem Benutzer hinterlegt sind, lassen den Nutzer auf unterschiedlichste Unterseiten gelangen. Sobald ein Antrag fertig erstellt worden ist, wird dieser mit allen eingegeben Informationen an die \textit{REST}-Schnittstelle gesendet, damit dort der Antrag gespeichert wird.\\

Trotzdem, gibt es einige Verbesserungen, die in Zukunft umgesetzt werden könnten. Diese sind in der folgenden Liste aufgezählt:
\begin{itemize}
	\item Google-Maps Schnittstelle zur Berechnung der Fahrtkilometer
	\item Überprüfung, ob ein ähnlicher Antrag bereits existiert
	\item Vorlagen von Veranstaltungen speichern
	\item Verbesserung der Auswahl der Teilnehmenden Lehrer einer Schulveranstaltung
	\item Verbesserung der Fehlerbehandlung der Anfragen an die \textit{REST}-Schnittstelle
	\item Verbesserung der allgemeinen Performanz der Datenschnittstelle sowie der Webseitenlogik
\end{itemize}
\captionof{listing}[Datenschnittstelle und Webseitenlogik Verbesserungsideen]{Auflistungen von Verbesserungen zur zukünftigen Implementierung, der Datenschnittstelle und Webseitenlogik}