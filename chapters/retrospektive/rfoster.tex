\section{Retrospektive Datenschnittstelle und Webseitenlogik}
Die Zusammenarbeit zwischen den Teammitgliedern war aufgrund der derzeitigen Pandemie stark beeinträchtigt, wodurch es neue Herausforderungen zu meistern gab. Durch diese Situation wurde die Entwicklung der Datenschnittstelle deutlich erschwert. Durch fehlende Informationen konnte die Datenschnittstelle erst sehr spät angefangen werden entwickelt zu werden. Auch die Entwicklung der Webseitenlogik hat durch fehlende Kommunikation länger gedauert, als geplant. Dadurch wurden Fehler in der Navigation sowie bei den Eingaben des Nutzers einprogrammiert, welche später behoben werden mussten. Daraus wurde gelernt, dass die Kommunikation ein essenzieller Teil eines Projektes ist, welcher nicht vernachlässigt werden darf. Durch herausragende Zusammenarbeit gegen Ende des Projekts konnten jedoch alle Herausforderungen bewältigt werden.\\
Viele Erfahrungen wurden während des Projektes in Verbindung mit Webdesign und Datenschnittstellen gesammelt. Durch intensive Auseinandersetzung mit der Themenbereich konnten einige neuen Fähigkeiten erlernt und angewendet werden. Die Kommunikation mit dem Backend war besonders interessant, da hier Arten von Abfragen zum Einsatz kamen, welche vorher noch nicht eingesetzt worden sind.