%!TEX root=../../main.tex
\section{Datenschnittstelle und Webseitenlogik}
\subsection{Webseitenlogik}
Wie in Kapitel \hyperref[sec:webseitenlogik]{6.3.1} beschrieben wird die Webseitenlogik größtenteils von einem Manager übernommen.

\subsubsection{Navigation}
Die Unterseiten werden in den Manager geladen und je nach Gebrauch angezeigt. Dafür gibt es eine Variable im Manager, welche dafür sorgt, dass nur eine Komponente auf einmal angezeigt werden kann. Die Komponente wird gewechselt, indem die nachfolgende Funktion aufgerufen wird:
\begin{code}{js}
	changeComponent(
	component,
	back = true,
	application = null,
	escortsdata = null
	) {
		switch (component) {
			case "Login":
			this.change("Login", back);
			this.deleteCookie();
			break;
			
			case "Index":
			this.change("Index", back);
			break;
			
			case "ApplicationView":
			this.loadApplication(application);
			this.change("ApplicationView", back);
			break;
			
			case "Escorts":
			this.loadEscortsData(escortsdata);
			this.change("Escorts", back, false);
			break;
			
			// [Weitere Unterseiten]
		}
	}
\end{code}
\captionof{listing}{Funktion zum ändern der angezeigten Komponente}~\\
\newpage
Diese Funktion besteht hauptsächlich aus einer Verzweigung, welche Seite angezeigt werden soll.
Des weiteren gibt es bei manchen Komponenten den Zusatz einer load Funktion, welche Daten für die jeweilige Komponente vorbereitet bzw. berechnet.

Die referenzierte change Funktion übernimmt das tatsächliche ändern der Variable:
\begin{code}{js}
	change(page, back = true, cookie = true) {
		this.currentComponent = page;	// Es wird die angezeigte Seite verändert
		window.scrollTo(0, 0);	// Es wird zum Anfang der Seite gegangen
		if (back) {
			if (window.history.state !== page) {
				window.history.pushState(page, null);	// Es wird die übergebene Seite in die History des Browsers geschrieben
			}
		}
		if (cookie) {
			this.setCookie(page);	// Es wird der Cookie gesetzt
		}
\end{code}
\captionof{listing}{Funktion für das Managment der Cookies und History des Browsers}~\\
Hier wird nicht nur die angezeigte Seite verändert, sondern auch sich um das Cookie-Management, sowie das History-Management gekümmert. Es wird je nach Parameter der Cookie und bzw. oder die History gesetzt.\\
Die beschriebenen Funktionen müssen von den Komponenten aufgerufen werden können, deswegen hört der Manager auf Signale der Unterseiten, falls diese die Seite ändern möchte. Dies wird mittels VueJS erledigt, da das Framework dafür bereits Funktionen implementiert hat:
\begin{code}{js}
	v-on:change-component="changeComponent"
	// Es wird auf das Signal change-component gehört und die Funktion changeComponent aufgerufen
\end{code}
\captionof{listing}{Event Handling mittels v-on}~\\
Die Codezeile wird beim Aufruf der Unterseite im Manager hinzugefügt.\\
Die beschriebenen Signale werden von den Unterseiten mittels folgendem Code bzw. Funktion gesendet:
\begin{code}{js}
	changeComponent(component, back, application, escortsdata) {
		this.$emit("change-component", component, back, application, escortsdata);
	}
\end{code}
\captionof{listing}{Signal senden}~\\
Hier wird durch this.\$emit() ein Befehl an die übergeordnete Seite gesendet.
\subsubsection{Komponeten}
\subsubsection{Seite nicht gefunden}
\subsubsection{Antrag-Viewer}
\subsubsection{Features}
\subsection{Daten laden}
\subsection{Befehle senden}
